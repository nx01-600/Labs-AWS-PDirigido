\documentclass[12pt,a4paper]{article}

% Paquetes necesarios
\usepackage[utf8]{inputenc}
\usepackage[spanish]{babel}
\usepackage{graphicx}
\usepackage{listings}
\usepackage{xcolor}
\usepackage{hyperref}
\usepackage{geometry}
\usepackage{fancyhdr}
\usepackage{titlesec}
\usepackage{enumitem}
\usepackage{float}
\usepackage{caption}

% Configuración de página
\geometry{
    left=2.5cm,
    right=2.5cm,
    top=3cm,
    bottom=3cm
}

% Configuración de encabezado y pie de página
\pagestyle{fancy}
\fancyhf{}
\fancyhead[L]{Laboratorios Virtuales de Redes en AWS}
\fancyhead[R]{Lab \#X}
\fancyfoot[C]{\thepage}

% Configuración de hipervínculos
\hypersetup{
    colorlinks=true,
    linkcolor=blue,
    filecolor=magenta,      
    urlcolor=cyan,
    pdftitle={Laboratorio X - AWS Redes},
    pdfauthor={Nicolás Carreño Tascón, Juan Manuel Canchala Jiménez},
}

% Configuración de código
\lstset{
    backgroundcolor=\color{gray!10},
    basicstyle=\ttfamily\small,
    breaklines=true,
    captionpos=b,
    commentstyle=\color{green!60!black},
    keywordstyle=\color{blue},
    stringstyle=\color{orange},
    showstringspaces=false,
    numbers=left,
    numberstyle=\tiny\color{gray},
    frame=single,
    rulecolor=\color{gray!30},
    tabsize=2
}

% Configuración de títulos de sección
\titleformat{\section}
{\normalfont\Large\bfseries\color{blue!70!black}}
{\thesection}{1em}{}

\titleformat{\subsection}
{\normalfont\large\bfseries\color{blue!50!black}}
{\thesubsection}{1em}{}

\begin{document}

% ================== PORTADA ==================
\begin{titlepage}
    \centering
    \includegraphics[width=0.3\textwidth]{logo_universidad.png}\\[1cm] % Agregar logo si existe
    
    \vspace{1cm}
    {\huge\bfseries Laboratorio \#X\par}
    \vspace{0.5cm}
    {\Large\bfseries [TÍTULO DEL LABORATORIO]\par}
    \vspace{2cm}
    
    {\large\textbf{Proyecto:}\par}
    {\large Laboratorios Virtuales de Redes en AWS para el\par}
    {\large Fortalecimiento de Competencias en Redes de Nueva Generación\par}
    \vspace{1.5cm}
    
    {\large\textbf{Estudiantes:}\par}
    {\large Nicolás Carreño Tascón\par}
    {\large Juan Manuel Canchala Jiménez\par}
    \vspace{1cm}
    
    {\large\textbf{Director:}\par}
    {\large Carlos Olarte\par}
    \vspace{1.5cm}
    
    {\large\textbf{Asignatura:}\par}
    {\large Redes de Nueva Generación\par}
    \vspace{1cm}
    
    {\large\textbf{Universidad:}\par}
    {\large [Nombre de la Universidad]\par}
    \vspace{1cm}
    
    {\large \today\par}
\end{titlepage}

% ================== TABLA DE CONTENIDOS ==================
\tableofcontents
\newpage

% ================== RESUMEN ==================
\section*{Resumen}
\addcontentsline{toc}{section}{Resumen}

[Breve descripción del laboratorio, qué se aprenderá y por qué es importante]

\vspace{0.5cm}
\noindent\textbf{Palabras clave:} AWS, VPC, Redes, [otros términos relevantes]

\newpage

% ================== OBJETIVOS ==================
\section{Objetivos}

\subsection{Objetivo General}
[Describir el objetivo principal del laboratorio]

\subsection{Objetivos Específicos}
\begin{itemize}
    \item [Objetivo específico 1]
    \item [Objetivo específico 2]
    \item [Objetivo específico 3]
    \item [Objetivo específico 4]
\end{itemize}

\subsection{Competencias a Desarrollar}
\begin{itemize}
    \item [Competencia 1]
    \item [Competencia 2]
    \item [Competencia 3]
\end{itemize}

% ================== MARCO TEÓRICO ==================
\section{Marco Teórico}

\subsection{Conceptos Fundamentales}
[Explicación de los conceptos teóricos necesarios]

\subsection{Arquitectura y Componentes}
[Diagramas y explicación de la arquitectura]

\begin{figure}[H]
    \centering
    % \includegraphics[width=0.8\textwidth]{diagrama.png}
    \caption{Diagrama de arquitectura del laboratorio}
    \label{fig:arquitectura}
\end{figure}

\subsection{Casos de Uso}
[Casos de uso reales donde se aplica lo aprendido]

% ================== REQUISITOS PREVIOS ==================
\section{Requisitos Previos}

\subsection{Conocimientos Necesarios}
\begin{itemize}
    \item [Conocimiento 1]
    \item [Conocimiento 2]
    \item [Conocimiento 3]
\end{itemize}

\subsection{Recursos AWS Requeridos}
\begin{itemize}
    \item Cuenta de AWS (Free Tier es suficiente)
    \item [Otros servicios necesarios]
\end{itemize}

\subsection{Costos Estimados}
\begin{itemize}
    \item \textbf{Con Free Tier:} \$0.00 (si se siguen las instrucciones)
    \item \textbf{Sin Free Tier:} Aproximadamente \$X.XX por hora
    \item \textbf{Recomendación:} Eliminar recursos al finalizar
\end{itemize}

\subsection{Tiempo Estimado}
\begin{itemize}
    \item Lectura y preparación: XX minutos
    \item Implementación: XX minutos
    \item Verificación y pruebas: XX minutos
    \item \textbf{Total:} XX-XX minutos
\end{itemize}

% ================== PROCEDIMIENTO ==================
\section{Procedimiento Paso a Paso}

\subsection{Paso 1: [Título del paso]}
\textbf{Descripción:} [Explicación de qué se hará en este paso]

\textbf{Instrucciones:}
\begin{enumerate}
    \item [Instrucción detallada 1]
    \item [Instrucción detallada 2]
    \item [Instrucción detallada 3]
\end{enumerate}

\begin{figure}[H]
    \centering
    % \includegraphics[width=0.9\textwidth]{paso1.png}
    \caption{Captura de pantalla del Paso 1}
    \label{fig:paso1}
\end{figure}

\textbf{Código/Configuración:}
\begin{lstlisting}[language=bash, caption=Ejemplo de comando CLI]
# Comando de ejemplo
aws ec2 describe-vpcs
\end{lstlisting}

\subsection{Paso 2: [Título del paso]}
[Repetir estructura para cada paso...]

% ================== VERIFICACIÓN ==================
\section{Verificación y Pruebas}

\subsection{Verificar Configuración}
\textbf{Objetivo:} Confirmar que la configuración es correcta

\begin{enumerate}
    \item [Paso de verificación 1]
    \item [Paso de verificación 2]
    \item [Paso de verificación 3]
\end{enumerate}

\subsection{Pruebas de Conectividad}
\textbf{Prueba 1:} [Descripción]
\begin{lstlisting}[language=bash]
# Comando de prueba
ping 10.0.1.10
\end{lstlisting}

\textbf{Resultado esperado:} [Qué se debe observar]

\subsection{Troubleshooting}
\textbf{Problema común 1:} [Descripción del problema]
\begin{itemize}
    \item \textbf{Causa:} [Explicación]
    \item \textbf{Solución:} [Cómo resolverlo]
\end{itemize}

% ================== LIMPIEZA ==================
\section{Limpieza de Recursos}

\textbf{IMPORTANTE:} Para evitar cargos no deseados, elimina todos los recursos creados.

\subsection{Pasos de Limpieza}
\begin{enumerate}
    \item [Paso de limpieza 1]
    \item [Paso de limpieza 2]
    \item [Paso de limpieza 3]
\end{enumerate}

\subsection{Verificar Eliminación}
\begin{lstlisting}[language=bash]
# Verificar que no quedan recursos
aws ec2 describe-instances --filters "Name=instance-state-name,Values=running"
\end{lstlisting}

% ================== ACTIVIDADES COMPLEMENTARIAS ==================
\section{Actividades Complementarias}

\subsection{Preguntas de Reflexión}
\begin{enumerate}
    \item [Pregunta conceptual 1]
    \item [Pregunta práctica 2]
    \item [Pregunta de diseño 3]
\end{enumerate}

\subsection{Ejercicios Adicionales}
\begin{enumerate}
    \item \textbf{Ejercicio 1:} [Descripción del ejercicio adicional]
    \item \textbf{Ejercicio 2:} [Otro ejercicio de ampliación]
\end{enumerate}

\subsection{Reto Avanzado}
[Proponer un reto más complejo relacionado con el tema]

% ================== CONCLUSIONES ==================
\section{Conclusiones}

\begin{itemize}
    \item [Conclusión 1]
    \item [Conclusión 2]
    \item [Conclusión 3]
\end{itemize}

% ================== REFERENCIAS ==================
\section{Referencias y Recursos Adicionales}

\subsection{Documentación Oficial de AWS}
\begin{itemize}
    \item \href{https://docs.aws.amazon.com/vpc/}{Amazon VPC Documentation}
    \item \href{https://aws.amazon.com/architecture/}{AWS Architecture Center}
    \item [Otros enlaces relevantes]
\end{itemize}

\subsection{Tutoriales y Guías}
\begin{itemize}
    \item [Tutorial 1]
    \item [Tutorial 2]
\end{itemize}

\subsection{Videos Recomendados}
\begin{itemize}
    \item [Video 1]
    \item [Video 2]
\end{itemize}

% ================== ANEXOS ==================
\newpage
\appendix
\section{Anexo A: Comandos Útiles}

\begin{lstlisting}[language=bash, caption=Colección de comandos útiles]
# Listar VPCs
aws ec2 describe-vpcs

# Listar subnets
aws ec2 describe-subnets

# Listar instancias
aws ec2 describe-instances
\end{lstlisting}

\section{Anexo B: Diagramas Adicionales}
[Diagramas complementarios si son necesarios]

\section{Anexo C: Glosario de Términos}
\begin{itemize}
    \item \textbf{VPC:} Virtual Private Cloud
    \item \textbf{CIDR:} Classless Inter-Domain Routing
    \item [Otros términos...]
\end{itemize}

\end{document}
