\documentclass[12pt,a4paper]{article}

% Paquetes necesarios
\usepackage[utf8]{inputenc}
\usepackage[spanish]{babel}
\usepackage{graphicx}
\usepackage{listings}
\usepackage{xcolor}
\usepackage{hyperref}
\usepackage{geometry}
\usepackage{fancyhdr}
\usepackage{titlesec}
\usepackage{enumitem}
\usepackage{float}
\usepackage{caption}
\usepackage{tikz}
\usetikzlibrary{shapes.geometric, arrows, positioning}

% Configuración de página
\geometry{
    left=2.5cm,
    right=2.5cm,
    top=3cm,
    bottom=3cm
}

% Configuración de encabezado y pie de página
\pagestyle{fancy}
\fancyhf{}
\fancyhead[L]{Laboratorios Virtuales de Redes en AWS}
\fancyhead[R]{Lab \#8}
\fancyfoot[C]{\thepage}

% Configuración de hipervínculos
\hypersetup{
    colorlinks=true,
    linkcolor=blue,
    filecolor=magenta,      
    urlcolor=cyan,
    pdftitle={Laboratorio 8 - Proyecto Integrador - Arquitectura Completa},
    pdfauthor={Nicolás Carreño Tascón, Juan Manuel Canchala Jiménez},
}

% Configuración de código
\lstset{
    backgroundcolor=\color{gray!10},
    basicstyle=\ttfamily\small,
    breaklines=true,
    captionpos=b,
    commentstyle=\color{green!60!black},
    keywordstyle=\color{blue},
    stringstyle=\color{orange},
    showstringspaces=false,
    numbers=left,
    numberstyle=\tiny\color{gray},
    frame=single,
    rulecolor=\color{gray!30},
    tabsize=2
}

% Configuración de títulos
\titleformat{\section}
{\normalfont\Large\bfseries\color{blue!70!black}}
{\thesection}{1em}{}

\titleformat{\subsection}
{\normalfont\large\bfseries\color{blue!50!black}}
{\thesubsection}{1em}{}

\begin{document}

% ================== PORTADA ==================
\begin{titlepage}
    \centering
    \vspace{2cm}
    {\huge\bfseries Laboratorio \#8\par}
    \vspace{0.5cm}
    {\Large\bfseries Proyecto Integrador - Arquitectura Completa en AWS\par}
    \vspace{2cm}
    
    {\large\textbf{Proyecto:}\par}
    {\large Laboratorios Virtuales de Redes en AWS para el\par}
    {\large Fortalecimiento de Competencias en Redes de Nueva Generación\par}
    \vspace{1.5cm}
    
    {\large\textbf{Estudiantes:}\par}
    {\large Nicolás Carreño Tascón\par}
    {\large Juan Manuel Canchala Jiménez\par}
    \vspace{1cm}
    
    {\large\textbf{Director:}\par}
    {\large Carlos Olarte\par}
    \vspace{1.5cm}
    
    {\large\textbf{Asignatura:}\par}
    {\large Redes de Nueva Generación\par}
    \vspace{1cm}
    
    {\large\textbf{Duración Estimada:} 180 minutos\par}
    {\large\textbf{Costo:} \$0.00 (100\% Gratuito - Free Tier)\par}
    \vspace{1cm}
    
    {\large Diciembre 2025\par}
\end{titlepage}

% ================== TABLA DE CONTENIDOS ==================
\tableofcontents
\newpage

% ================== RESUMEN ==================
\section*{Resumen}
\addcontentsline{toc}{section}{Resumen}

Este laboratorio constituye el \textbf{proyecto integrador final} de la serie de laboratorios de redes en AWS. El objetivo es diseñar e implementar una \textbf{arquitectura corporativa completa} en la nube, integrando de forma coherente todos los conceptos trabajados en los laboratorios anteriores: creación de cuenta e IAM (Lab 1), diseño de VPC y subredes (Lab 2), conectividad a internet con Internet Gateway (Lab 3), instancias EC2 y seguridad de red (Lab 4), seguridad avanzada y defensa en profundidad (Lab 5), VPC Peering (Lab 6) y monitoreo con CloudWatch y VPC Flow Logs (Lab 7).

El estudiante diseñará una red corporativa simplificada para una pequeña-mediana empresa, que incluya una VPC principal con subredes públicas y privadas distribuidas en múltiples zonas de disponibilidad, instancias EC2 para capa web y de administración, grupos de seguridad multicapa, NACLs, conectividad a internet mediante IGW, monitoreo con CloudWatch y VPC Flow Logs, y un esquema teórico de alta disponibilidad y escalabilidad. Se documentarán explícitamente las \textbf{decisiones arquitectónicas}, se realizará un \textbf{análisis de costos y optimización dentro del Free Tier} y se presentará la solución de forma profesional, como si se tratara de una propuesta para un cliente real.

La implementación práctica se limitará a un subconjunto mínimo de recursos para mantener el costo en \$0.00 (Free Tier), mientras que la arquitectura completa se describirá a nivel de diseño lógico y buenas prácticas.

\vspace{0.5cm}
\noindent\textbf{Palabras clave:} AWS, Arquitectura de Red, VPC, EC2, Security Groups, VPC Peering, CloudWatch, VPC Flow Logs, Alta Disponibilidad, Free Tier.

\newpage

% ================== OBJETIVOS ==================
\section{Objetivos}

\subsection{Objetivo General}

Diseñar y documentar una \textbf{arquitectura corporativa completa en AWS}, integrando todos los componentes de red trabajados en los laboratorios 1–7, junto con una implementación básica que se mantenga dentro del Free Tier y refleje buenas prácticas de seguridad, escalabilidad y monitoreo.

\subsection{Objetivos Específicos}

\begin{itemize}[leftmargin=*]
    \item Integrar VPC, subredes públicas y privadas, Internet Gateway, EC2, Security Groups, VPC Peering y CloudWatch en una arquitectura coherente.
    \item Documentar de forma explícita las \textbf{decisiones arquitectónicas} tomadas (rangos de direcciones, distribución por AZ, separación por capas, patrones de seguridad, etc.).
    \item Diseñar un \textbf{escenario de alta disponibilidad y escalabilidad} (a nivel teórico) basado en el uso de múltiples zonas de disponibilidad y balanceo de carga.
    \item Realizar un \textbf{análisis de costos} de la arquitectura propuesta, identificando qué se mantiene en Free Tier y qué componentes tendrían costo en producción.
    \item Implementar de forma práctica una versión mínima de la arquitectura (al menos una VPC, subred pública y privada, 1–2 instancias EC2 y seguridad básica) y verificar su funcionamiento.
    \item Aplicar mecanismos de \textbf{monitoreo y registro} (VPC Flow Logs, métricas y alarmas de CloudWatch) sobre la arquitectura básica implementada.
    \item Presentar la solución final con un nivel de detalle y formalidad similar al de un \textbf{diseño profesional} de red corporativa en la nube.
\end{itemize}

\subsection{Competencias a Desarrollar}

\begin{itemize}[leftmargin=*]
    \item \textbf{Diseño arquitectónico en la nube:} Capacidad para diseñar topologías de red completas en AWS considerando seguridad, disponibilidad y monitoreo.
    \item \textbf{Toma de decisiones técnicas:} Selección y justificación de componentes de AWS adecuados a los requisitos funcionales y no funcionales.
    \item \textbf{Análisis de costo-beneficio:} Evaluación de alternativas técnicas a la luz de su impacto económico (Free Tier vs producción).
    \item \textbf{Documentación profesional:} Elaboración de documentos de diseño claros, estructurados y orientados a clientes o equipos técnicos.
    \item \textbf{Operación y monitoreo:} Configuración de registros, métricas y alarmas para tener visibilidad del comportamiento de la red y los servicios.
\end{itemize}

\newpage

% ================== MARCO TEÓRICO ==================
\section{Marco Teórico}

\subsection{Arquitectura de Red Corporativa en AWS}

Una \textbf{red corporativa en AWS} suele estructurarse siguiendo patrones de arquitectura bien conocidos:

\begin{itemize}[leftmargin=*]
    \item \textbf{Separación por capas (n-tier):}
    \begin{itemize}
        \item Capa de presentación (web).
        \item Capa de lógica de negocio (aplicación).
        \item Capa de datos (bases de datos).
    \end{itemize}
    \item \textbf{Separación por dominios de seguridad:}
    \begin{itemize}
        \item Subredes públicas: recursos expuestos a internet (por ejemplo, balanceadores y bastions).
        \item Subredes privadas: recursos internos (aplicaciones, bases de datos).
    \end{itemize}
    \item \textbf{Distribución en múltiples zonas de disponibilidad (AZs):}
    \begin{itemize}
        \item Subredes replicadas en al menos 2 AZs para alta disponibilidad.
    \end{itemize}
\end{itemize}

La arquitectura integradora de este laboratorio se basará en una \textbf{VPC principal corporativa} con:
\begin{itemize}[leftmargin=*]
    \item CIDR base \texttt{10.0.0.0/16}.
    \item Subredes públicas y privadas en al menos dos AZs.
    \item Instancias EC2 en capa de presentación y administración.
    \item Reglas de seguridad multicapa (Security Groups + NACLs).
    \item Integración con CloudWatch para monitoreo.
\end{itemize}

\subsection{Resumen Integrado de Conceptos Anteriores}

\subsubsection{Cuenta, IAM y MFA (Lab 1)}

\begin{itemize}[leftmargin=*]
    \item \textbf{Cuenta AWS:} es el contenedor administrativo de todos los recursos.
    \item \textbf{IAM (Identity and Access Management):} controla quién puede hacer qué.
    \item \textbf{MFA:} protege la cuenta root y usuarios privilegiados.
    \item \textbf{Alertas de facturación:} sirven para evitar costos inesperados.
\end{itemize}

En el proyecto integrador, se asume:
\begin{itemize}[leftmargin=*]
    \item Usuario IAM administrativo con MFA habilitado.
    \item Alertas de facturación activas para mantener el costo en Free Tier.
\end{itemize}

\subsubsection{VPC, Subredes y Rutas (Lab 2 y Lab 3)}

\begin{itemize}[leftmargin=*]
    \item \textbf{VPC:} red virtual aislada con un rango CIDR.
    \item \textbf{Subredes:} divisiones lógicas del rango CIDR principal.
    \item \textbf{Internet Gateway (IGW):} proporciona acceso a internet desde y hacia la VPC.
    \item \textbf{Tablas de ruteo:} definen por dónde debe ir el tráfico.
\end{itemize}

La arquitectura integradora usará:
\begin{itemize}[leftmargin=*]
    \item Una VPC principal con cidr \texttt{10.0.0.0/16}.
    \item Subredes públicas (por ejemplo, \texttt{10.0.1.0/24} y \texttt{10.0.2.0/24}).
    \item Subredes privadas (por ejemplo, \texttt{10.0.11.0/24} y \texttt{10.0.12.0/24}).
    \item IGW para salida a internet desde subredes públicas.
\end{itemize}

\subsubsection{Instancias EC2 y Seguridad de Red (Lab 4 y Lab 5)}

\begin{itemize}[leftmargin=*]
    \item \textbf{EC2:} máquinas virtuales en la nube.
    \item \textbf{Security Groups (SG):} firewalls \textit{stateful} a nivel de instancia.
    \item \textbf{Network ACLs (NACLs):} listas de control de acceso \textit{stateless} a nivel de subred.
    \item \textbf{Defense in depth:} múltiples capas de seguridad (SG + NACL + IAM + monitoreo).
    \item \textbf{Principio de mínimo privilegio:} solo permitir el tráfico estricto necesario.
\end{itemize}

En el proyecto integrador se diseñarán:
\begin{itemize}[leftmargin=*]
    \item Un SG para servidores web.
    \item Un SG para instancias privadas (aplicación / base de datos).
    \item Un SG para bastion host (administración).
    \item NACLs que refuercen las políticas de tráfico por subred.
\end{itemize}

\subsubsection{VPC Peering (Lab 6)}

\begin{itemize}[leftmargin=*]
    \item Permite conectar dos VPCs de forma privada usando la red de AWS.
    \item \textbf{No es transitivo:} si VPC A está conectada con B, y B con C, A no ve a C por defecto.
    \item Se usa para escenarios multi-VPC (por ejemplo, entorno \textit{producción} y \textit{gestión}).
\end{itemize}

En este laboratorio, el \textbf{peering se considerará a nivel de diseño} para un escenario donde la empresa tenga:
\begin{itemize}[leftmargin=*]
    \item Una VPC principal de producción.
    \item Una VPC de herramientas o administración conectada por peering (teórico).
\end{itemize}

\subsubsection{Monitoreo y VPC Flow Logs (Lab 7)}

\begin{itemize}[leftmargin=*]
    \item \textbf{CloudWatch:} métrica, logs, alarmas y dashboards.
    \item \textbf{VPC Flow Logs:} registro de tráfico ACCEPT/REJECT.
    \item \textbf{CloudWatch Logs Insights:} consultas sobre logs.
    \item \textbf{Alarmas:} condiciones que generan notificaciones.
\end{itemize}

En la arquitectura integradora:
\begin{itemize}[leftmargin=*]
    \item Se habilitarán Flow Logs sobre la VPC o subredes principales.
    \item Se configurarán métricas derivadas (por ejemplo, conexiones REJECT).
    \item Se integrarán alarmas clave en un dashboard de monitoreo de red.
\end{itemize}

\subsection{Diseño de Alta Disponibilidad y Escalabilidad (Teórico)}

Aunque la implementación práctica se limitará a pocos recursos por Free Tier, el diseño considerará:

\begin{itemize}[leftmargin=*]
    \item \textbf{Alta disponibilidad (HA):}
    \begin{itemize}
        \item Subredes replicadas en al menos dos AZs.
        \item Posible uso de balanceadores de carga (teórico) para distribuir tráfico.
    \end{itemize}
    \item \textbf{Escalabilidad:}
    \begin{itemize}
        \item Escalado horizontal (más instancias EC2 detrás de un Load Balancer).
        \item Auto Scaling Groups (ASG) a nivel teórico.
    \end{itemize}
    \item \textbf{Resiliencia:}
    \begin{itemize}
        \item Diseñar para que la falla de una AZ no implique caída total.
        \item Uso de servicios gestionados (por ejemplo, RDS Multi-AZ en un escenario real).
    \end{itemize}
\end{itemize}

\subsection{Análisis de Costos y Optimización en Free Tier}

Un elemento clave en la arquitectura es diferenciar entre:

\begin{itemize}[leftmargin=*]
    \item \textbf{Recursos efectivamente usados en el laboratorio:}
    \begin{itemize}
        \item 1–2 instancias EC2 t2.micro/t3.micro (bajo límite de horas).
        \item VPC, subredes, IGW, NACLs, SGs (sin costo).
        \item VPC Flow Logs y CloudWatch con uso moderado.
    \end{itemize}
    \item \textbf{Recursos que quedarían solo a nivel de diseño teórico:}
    \begin{itemize}
        \item Balanceador de carga (ALB).
        \item Auto Scaling Groups.
        \item NAT Gateway (tiene costo, se deja solo como teoría).
        \item Transit Gateway para multi-VPC complejas.
    \end{itemize}
\end{itemize}

El diseño debe \textbf{separar claramente} lo que se implementa en la práctica sin costo y lo que sería parte de una versión ``full'' en producción con presupuesto asignado.

\newpage

% ================== REQUISITOS PREVIOS ==================
\section{Requisitos Previos}

\subsection{Conocimientos Necesarios}

\begin{itemize}[leftmargin=*]
    \item Haber completado (o entendido) los laboratorios 1–7.
    \item Comprender:
    \begin{itemize}
        \item Conceptos de VPC, subredes, tablas de ruteo, IGW.
        \item Configuración básica de EC2 y acceso SSH.
        \item Conceptos de Security Groups y NACLs.
        \item Fundamentos de VPC Peering.
        \item Uso básico de CloudWatch, VPC Flow Logs y Logs Insights.
    \end{itemize}
\end{itemize}

\subsection{Recursos Técnicos}

\begin{itemize}[leftmargin=*]
    \item Cuenta AWS activa con Free Tier.
    \item Usuario IAM con permisos administrativos (o al menos sobre VPC, EC2, CloudWatch, logs, IAM de lectura).
    \item Navegador web moderno.
    \item Cliente SSH para conectarse a las instancias (PuTTY, OpenSSH, etc.).
\end{itemize}

\subsection{Costos Estimados}

\begin{table}[H]
\centering
\begin{tabular}{|l|c|}
\hline
\textbf{Concepto} & \textbf{Costo Estimado} \\
\hline
1--2 instancias EC2 t2.micro/t3.micro & \$0.00 (dentro de 750 horas/mes Free Tier) \\
\hline
VPC, subredes, IGW, tablas de ruteo & \$0.00 \\
\hline
Security Groups y NACLs & \$0.00 \\
\hline
VPC Peering (sin tráfico de datos) & \$0.00 \\
\hline
CloudWatch métricas básicas y hasta 10 alarmas & \$0.00 (Free Tier) \\
\hline
VPC Flow Logs (bajo volumen, uso limitado) & \$0.00 -- costo muy bajo controlado \\
\hline
\textbf{TOTAL (Laboratorio)} & \textbf{\$0.00} \\
\hline
\end{tabular}
\caption{Costos del Laboratorio 8 en modo práctico}
\end{table}

\subsection{Tiempo Estimado}

\begin{itemize}[leftmargin=*]
    \item Diseño conceptual de la arquitectura: 45 minutos.
    \item Implementación básica (VPC, subredes, IGW, EC2, SGs): 60–70 minutos.
    \item Configuración de Flow Logs, métricas y dashboard: 40–45 minutos.
    \item Verificación, limpieza y documentación final: 25–30 minutos.
    \item \textbf{TOTAL ESTIMADO:} 180 minutos.
\end{itemize}

\newpage

% ================== PROCEDIMIENTO PASO A PASO ==================
\section{Procedimiento Paso a Paso}

\subsection{Visión General del Proyecto}

Antes de iniciar, se define el \textbf{escenario corporativo}:

\begin{itemize}[leftmargin=*]
    \item Empresa ficticia: \textbf{AcmeCorp S.A.S.}
    \item Necesidades:
    \begin{itemize}
        \item Exponer un sitio web corporativo.
        \item Tener una capa de aplicación y datos protegida en subredes privadas.
        \item Administrar los servidores de forma segura mediante un bastion host.
        \item Monitorear el tráfico de red y detectar anomalías.
    \end{itemize}
\end{itemize}

\subsubsection*{Arquitectura Lógica (Descripción)}

\begin{itemize}[leftmargin=*]
    \item VPC principal \texttt{AcmeCorp-VPC} (\texttt{10.0.0.0/16}).
    \item Subredes públicas en dos AZs:
    \begin{itemize}
        \item \texttt{10.0.1.0/24} (Public-AZ1).
        \item \texttt{10.0.2.0/24} (Public-AZ2).
    \end{itemize}
    \item Subredes privadas en dos AZs:
    \begin{itemize}
        \item \texttt{10.0.11.0/24} (Private-AZ1).
        \item \texttt{10.0.12.0/24} (Private-AZ2).
    \end{itemize}
    \item IGW para salida/entrada a internet desde subredes públicas.
    \item 1 servidor web en subred pública (implementación práctica).
    \item 1 bastion host en subred pública para administración SSH.
    \item 1 servidor de aplicación (o datos) en subred privada.
    \item Security Groups específicos para cada rol.
    \item VPC Flow Logs habilitados para la VPC.
    \item Métricas y alarmas en CloudWatch.
\end{itemize}

\subsection{Paso 1: Diseño de Direccionamiento y Subredes}

\textbf{Objetivo:} Definir el plan de direccionamiento IP interno.

\begin{enumerate}[leftmargin=*]
    \item Elegir región (por ejemplo, \texttt{us-east-1} o \texttt{sa-east-1}).
    \item Definir CIDR de la VPC: \texttt{10.0.0.0/16}.
    \item Plan de subredes:
    \begin{itemize}
        \item Subred pública AZ1: \texttt{10.0.1.0/24}.
        \item Subred pública AZ2: \texttt{10.0.2.0/24}.
        \item Subred privada AZ1: \texttt{10.0.11.0/24}.
        \item Subred privada AZ2: \texttt{10.0.12.0/24}.
    \end{itemize}
    \item Reservar espacio extra para posibles subredes futuras (por ejemplo, \texttt{10.0.21.0/24} para administración).
\end{enumerate}

\subsection{Paso 2: Creación de la VPC y Subredes}

\textbf{Objetivo:} Implementar el diseño de VPC y subredes.

\begin{enumerate}[leftmargin=*]
    \item Ir a \textbf{VPC} $\rightarrow$ \textbf{Your VPCs} $\rightarrow$ \textbf{Create VPC}.
    \item Parámetros sugeridos:
    \begin{itemize}
        \item \textbf{Name}: \texttt{AcmeCorp-VPC}.
        \item \textbf{IPv4 CIDR}: \texttt{10.0.0.0/16}.
        \item Opciones adicionales: por defecto.
    \end{itemize}
    \item Crear subredes:
    \begin{itemize}
        \item \textbf{Subnet 1 (publica-AZ1)}:
        \begin{itemize}
            \item Name: \texttt{AcmeCorp-Public-AZ1}.
            \item VPC: \texttt{AcmeCorp-VPC}.
            \item AZ: seleccionar una (ej. \texttt{us-east-1a}).
            \item CIDR: \texttt{10.0.1.0/24}.
        \end{itemize}
        \item \textbf{Subnet 2 (publica-AZ2)}:
        \begin{itemize}
            \item Name: \texttt{AcmeCorp-Public-AZ2}.
            \item AZ diferente (ej. \texttt{us-east-1b}).
            \item CIDR: \texttt{10.0.2.0/24}.
        \end{itemize}
        \item \textbf{Subnet 3 (privada-AZ1)}:
        \begin{itemize}
            \item Name: \texttt{AcmeCorp-Private-AZ1}.
            \item CIDR: \texttt{10.0.11.0/24}.
        \end{itemize}
        \item \textbf{Subnet 4 (privada-AZ2)}:
        \begin{itemize}
            \item Name: \texttt{AcmeCorp-Private-AZ2}.
            \item CIDR: \texttt{10.0.12.0/24}.
        \end{itemize}
    \end{itemize}
    \item Marcar las subredes públicas para asignación automática de IPs públicas (opcional).
\end{enumerate}

\subsection{Paso 3: Internet Gateway y Tablas de Ruteo}

\textbf{Objetivo:} Proveer conectividad a internet a las subredes públicas.

\begin{enumerate}[leftmargin=*]
    \item Crear un Internet Gateway:
    \begin{itemize}
        \item En \textbf{VPC} $\rightarrow$ \textbf{Internet Gateways} $\rightarrow$ \textbf{Create internet gateway}.
        \item Name: \texttt{AcmeCorp-IGW}.
        \item Crear y luego \textbf{Attach to VPC} $\rightarrow$ seleccionar \texttt{AcmeCorp-VPC}.
    \end{itemize}
    \item Crear tabla de ruteo pública:
    \begin{itemize}
        \item \textbf{Route Tables} $\rightarrow$ \textbf{Create route table}.
        \item Name: \texttt{AcmeCorp-Public-RT}.
        \item VPC: \texttt{AcmeCorp-VPC}.
    \end{itemize}
    \item Agregar ruta por defecto a internet:
    \begin{itemize}
        \item Editar rutas.
        \item Destino: \texttt{0.0.0.0/0}.
        \item Target: \texttt{AcmeCorp-IGW}.
    \end{itemize}
    \item Asociar subredes públicas a esta tabla:
    \begin{itemize}
        \item Subredes: \texttt{AcmeCorp-Public-AZ1} y \texttt{AcmeCorp-Public-AZ2}.
    \end{itemize}
    \item Dejar la tabla de ruteo por defecto para subredes privadas (sin ruta a internet).
\end{enumerate}

\subsection{Paso 4: Security Groups y NACLs (Defensa en Profundidad)}

\textbf{Objetivo:} Definir políticas de seguridad multicapa.

\subsubsection*{Security Groups}

\begin{enumerate}[leftmargin=*]
    \item SG para servidor web:
    \begin{itemize}
        \item Name: \texttt{SG-Web}.
        \item Inbound:
        \begin{itemize}
            \item HTTP (80) desde \texttt{0.0.0.0/0} (para pruebas).
            \item SSH (22) \textbf{solo} desde IP pública del bastion (luego de creado) o desde la IP pública del administrador.
        \end{itemize}
        \item Outbound: \texttt{0.0.0.0/0} (por simplicidad en el lab).
    \end{itemize}
    \item SG para bastion host:
    \begin{itemize}
        \item Name: \texttt{SG-Bastion}.
        \item Inbound:
        \begin{itemize}
            \item SSH (22) solo desde la IP pública del estudiante/administrador.
        \end{itemize}
        \item Outbound:
        \begin{itemize}
            \item SSH (22) hacia subredes privadas (instancias internas).
        \end{itemize}
    \end{itemize}
    \item SG para servidor interno (aplicación/datos):
    \begin{itemize}
        \item Name: \texttt{SG-App}.
        \item Inbound:
        \begin{itemize}
            \item SSH (22) solo desde \texttt{SG-Bastion}.
            \item Puerto de aplicación (ej. 8080) solo desde \texttt{SG-Web} (teórico).
        \end{itemize}
        \item Outbound: \texttt{0.0.0.0/0} o restringido según necesidad.
    \end{itemize}
\end{enumerate}

\subsubsection*{NACLs (Opcional para refuerzo)}

\begin{enumerate}[leftmargin=*]
    \item NACL pública:
    \begin{itemize}
        \item Allow inbound HTTP(80)/SSH(22) desde internet (según sea necesario).
        \item Allow outbound respuestas establecidas.
    \end{itemize}
    \item NACL privada:
    \begin{itemize}
        \item Restringir el tráfico entrante a puertos específicos desde subredes públicas.
    \end{itemize}
\end{enumerate}

\subsection{Paso 5: Lanzar Instancias EC2 (Implementación Básica)}

\textbf{Objetivo:} Desplegar recursos mínimos de cómputo para probar la arquitectura.

\begin{enumerate}[leftmargin=*]
    \item \textbf{Servidor web:}
    \begin{itemize}
        \item Tipo: \texttt{t2.micro} o \texttt{t3.micro} (Free Tier).
        \item AMI: Amazon Linux 2 u otra Free Tier.
        \item Subred: \texttt{AcmeCorp-Public-AZ1}.
        \item Auto-assign Public IP: habilitado.
        \item SG: \texttt{SG-Web}.
        \item User data opcional: instalar un servidor web simple (ejemplo con \texttt{httpd}).
    \end{itemize}
    \item \textbf{Bastion host:}
    \begin{itemize}
        \item Tipo: \texttt{t2.micro}/\texttt{t3.micro}.
        \item Subred: \texttt{AcmeCorp-Public-AZ1} (o AZ2).
        \item IP pública asignada.
        \item SG: \texttt{SG-Bastion}.
    \end{itemize}
    \item \textbf{Servidor interno:}
    \begin{itemize}
        \item Tipo: \texttt{t2.micro}/\texttt{t3.micro}.
        \item Subred: \texttt{AcmeCorp-Private-AZ1}.
        \item Sin IP pública.
        \item SG: \texttt{SG-App}.
    \end{itemize}
\end{enumerate}

\subsection{Paso 6: Pruebas de Conectividad y Seguridad}

\begin{enumerate}[leftmargin=*]
    \item Desde el navegador del estudiante, acceder a la IP pública del servidor web vía HTTP.
    \item Conectarse por SSH al bastion host usando la clave privada.
    \item Desde el bastion host, hacer SSH al servidor interno (private IP).
    \item Verificar que no se pueda acceder directamente al servidor interno desde internet.
\end{enumerate}

\subsection{Paso 7: Habilitar VPC Flow Logs y Monitoreo}

\textbf{Objetivo:} Integrar monitoreo en la arquitectura.

\begin{enumerate}[leftmargin=*]
    \item Crear log group en CloudWatch: \texttt{/aws/vpc/flow-logs/acmecorp}.
    \item En VPC $\rightarrow$ \textbf{Your VPCs} $\rightarrow$ \texttt{AcmeCorp-VPC} $\rightarrow$ pestaña \textbf{Flow Logs}.
    \item Crear flow log con:
    \begin{itemize}
        \item Filter: \texttt{ALL}.
        \item Destination: CloudWatch Logs.
        \item Log group: \texttt{/aws/vpc/flow-logs/acmecorp}.
        \item Interval: 1 minuto.
    \end{itemize}
    \item Generar tráfico de prueba (HTTP permitido, puertos bloqueados).
    \item Verificar logs en CloudWatch.
\end{enumerate}

\subsection{Paso 8: Métricas Derivadas y Alarmas}

\begin{enumerate}[leftmargin=*]
    \item Crear un metric filter para \texttt{REJECT} similar al Lab 7:
    \begin{itemize}
        \item Log group: \texttt{/aws/vpc/flow-logs/acmecorp}.
        \item Filter pattern: \texttt{REJECT}.
        \item Namespace: \texttt{AcmeCorp/Network}.
        \item Metric name: \texttt{RejectedConnections}.
    \end{itemize}
    \item Crear una alarma \texttt{AcmeCorp-HighRejectedConnections} sobre esta métrica.
    \item Opcional: crear alarma sobre \texttt{NetworkIn} del servidor web para detectar picos de tráfico.
\end{enumerate}

\subsection{Paso 9: Dashboard de Arquitectura y Monitoreo}

\begin{enumerate}[leftmargin=*]
    \item Crear dashboard \texttt{AcmeCorp-Network-Dashboard}.
    \item Agregar widgets:
    \begin{itemize}
        \item Gráfico de \texttt{NetworkIn/Out} del servidor web.
        \item Gráfico de \texttt{RejectedConnections}.
        \item Widget de estado de alarmas.
        \item Opcional: un widget de texto con un diagrama ASCII simplificado de la arquitectura.
    \end{itemize}
\end{enumerate}

\subsection{Paso 10: Diseño Teórico de Escalabilidad y Alta Disponibilidad}

Este paso es \textbf{solo teórico} (no implementar por costo):

\begin{itemize}[leftmargin=*]
    \item Proponer el uso de:
    \begin{itemize}
        \item Application Load Balancer (ALB) frente a una flota de servidores web.
        \item Auto Scaling Group con mínimo 2 instancias en 2 AZs.
        \item NAT Gateway (o NAT instance) para permitir salida a internet desde subredes privadas (teniendo en cuenta el costo).
        \item Posible segunda VPC (administración) conectada por VPC Peering.
    \end{itemize}
    \item Documentar cómo cambiaría la arquitectura básica para llegar a una solución de alta disponibilidad en producción.
\end{itemize}

\newpage

% ================== TABLAS DE CONFIGURACIÓN ==================
\section{Tablas de Configuración}

\subsection{Plan de Direccionamiento}

\begin{table}[H]
\centering
\begin{tabular}{|l|l|l|}
\hline
\textbf{Recurso} & \textbf{Nombre} & \textbf{CIDR} \\
\hline
VPC & AcmeCorp-VPC & 10.0.0.0/16 \\
\hline
Subred pública AZ1 & AcmeCorp-Public-AZ1 & 10.0.1.0/24 \\
\hline
Subred pública AZ2 & AcmeCorp-Public-AZ2 & 10.0.2.0/24 \\
\hline
Subred privada AZ1 & AcmeCorp-Private-AZ1 & 10.0.11.0/24 \\
\hline
Subred privada AZ2 & AcmeCorp-Private-AZ2 & 10.0.12.0/24 \\
\hline
\end{tabular}
\caption{Plan de direccionamiento de la arquitectura integradora}
\end{table}

\subsection{Resumen de Componentes Clave}

\begin{table}[H]
\centering
\begin{tabular}{|l|l|p{7cm}|}
\hline
\textbf{Componente} & \textbf{Nombre} & \textbf{Descripción} \\
\hline
VPC & AcmeCorp-VPC & Red corporativa principal \\
\hline
IGW & AcmeCorp-IGW & Conectividad a internet \\
\hline
RT Pública & AcmeCorp-Public-RT & Tabla de rutas para subredes públicas \\
\hline
SG & SG-Web & Seguridad para servidor web \\
\hline
SG & SG-Bastion & Seguridad para bastion host \\
\hline
SG & SG-App & Seguridad para servidor interno \\
\hline
Flow Logs & AcmeCorp-VPC Flow Log & Registros de tráfico de la VPC \\
\hline
Log Group & /aws/vpc/flow-logs/acmecorp & Almacenamiento de VPC Flow Logs \\
\hline
Namespace & AcmeCorp/Network & Métricas personalizadas de red \\
\hline
Métrica & RejectedConnections & Conteo de conexiones REJECT \\
\hline
Alarma & AcmeCorp-HighRejectedConnections & Alerta por tráfico rechazado elevado \\
\hline
Dashboard & AcmeCorp-Network-Dashboard & Panel de monitoreo integrado \\
\hline
\end{tabular}
\caption{Componentes principales del proyecto integrador}
\end{table}

\newpage

% ================== VERIFICACIÓN ==================
\section{Verificación}

\subsection{Pruebas de Conectividad}

\begin{itemize}[leftmargin=*]
    \item \textbf{Prueba 1: Acceso web}
    \begin{itemize}
        \item Desde el navegador local, acceder a \texttt{http://IP-publica-servidor-web}.
        \item Verificar que responde la página por defecto o el contenido configurado.
    \end{itemize}
    \item \textbf{Prueba 2: Acceso SSH seguro}
    \begin{itemize}
        \item Conectarse por SSH al bastion host usando clave privada.
        \item Desde el bastion, conectarse por SSH a la IP privada del servidor interno.
    \end{itemize}
    \item \textbf{Prueba 3: Aislamiento del servidor interno}
    \begin{itemize}
        \item Intentar conectarse directamente desde internet al servidor interno (debe fallar).
        \item Confirmar que solo es accesible a través del bastion.
    \end{itemize}
\end{itemize}

\subsection{Verificación de Flow Logs y Métricas}

\begin{enumerate}[leftmargin=*]
    \item Verificar que el Flow Log de la VPC está en estado \textbf{Active}.
    \item Revisar el log group \texttt{/aws/vpc/flow-logs/acmecorp} para ver registros recientes.
    \item Confirmar que la métrica \texttt{RejectedConnections} se está actualizando en \texttt{AcmeCorp/Network}.
\end{enumerate}

\subsection{Verificación de Alarmas y Dashboard}

\begin{enumerate}[leftmargin=*]
    \item Verificar en \textbf{CloudWatch Alarms} el estado de \texttt{AcmeCorp-HighRejectedConnections}.
    \item Forzar algunos intentos de conexión bloqueados (por ejemplo, puertos cerrados) y observar si la métrica aumenta.
    \item Revisar el dashboard \texttt{AcmeCorp-Network-Dashboard} y comprobar que:
    \begin{itemize}
        \item Se muestra el tráfico de red del servidor web.
        \item Se visualiza la curva de \texttt{RejectedConnections}.
        \item Se ve el widget con el estado de alarmas.
    \end{itemize}
\end{enumerate}

\newpage

% ================== LIMPIEZA DE RECURSOS ==================
\section{Limpieza de Recursos}

\textbf{Objetivo:} Dejar la cuenta libre de recursos que puedan generar costos más allá del laboratorio.

\begin{enumerate}[leftmargin=*]
    \item \textbf{Instancias EC2:}
    \begin{itemize}
        \item Detener y terminar el servidor web, bastion host y servidor interno si no se van a seguir usando.
    \end{itemize}
    \item \textbf{Flow Logs:}
    \begin{itemize}
        \item En la VPC, eliminar el Flow Log configurado para detener la generación de logs.
    \end{itemize}
    \item \textbf{CloudWatch Logs:}
    \begin{itemize}
        \item Eliminar el log group \texttt{/aws/vpc/flow-logs/acmecorp} si ya no es necesario.
    \end{itemize}
    \item \textbf{Alarmas y métricas:}
    \begin{itemize}
        \item Eliminar la alarma \texttt{AcmeCorp-HighRejectedConnections}.
        \item (Opcional) Eliminar el metric filter asociado.
    \end{itemize}
    \item \textbf{Dashboard:}
    \begin{itemize}
        \item Eliminar el dashboard \texttt{AcmeCorp-Network-Dashboard} si no se usará más.
    \end{itemize}
    \item \textbf{VPC y redes:}
    \begin{itemize}
        \item Eliminar subredes, tablas de ruteo personalizadas, IGW y finalmente la VPC \texttt{AcmeCorp-VPC} si fue creada exclusivamente para el laboratorio.
    \end{itemize}
\end{enumerate}

\newpage

% ================== CUESTIONARIO INTEGRADO ==================
\section{Cuestionario de Evaluación}

\subsection{Preguntas de Selección Múltiple}

\begin{enumerate}

\item \textbf{¿Cuál es el principal objetivo del proyecto integrador del Lab 8?}
\begin{enumerate}[label=\alph*)]
    \item Probar únicamente el rendimiento de instancias EC2.
    \item Diseñar una arquitectura corporativa completa integrando los conceptos de los labs 1–7.
    \item Configurar NAT Gateway en producción.
    \item Migrar una base de datos on-premise a AWS RDS.
\end{enumerate}

\item \textbf{En la arquitectura propuesta, las subredes privadas se utilizan principalmente para:}
\begin{enumerate}[label=\alph*)]
    \item Alojar recursos expuestos directamente a internet.
    \item Almacenar objetos S3 con acceso público.
    \item Instancias internas de aplicación y bases de datos sin IP pública.
    \item Consolas de administración web públicas.
\end{enumerate}

\item \textbf{¿Cuál de los siguientes componentes NO genera costo directo en el Free Tier (uso razonable)?}
\begin{enumerate}[label=\alph*)]
    \item 1–2 instancias EC2 t2.micro/t3.micro dentro de las 750 horas/mes.
    \item VPC, subredes, tablas de ruteo e IGW.
    \item NAT Gateway con tráfico de salida a internet.
    \item Security Groups y NACLs.
\end{enumerate}

\item \textbf{En el diseño de defensa en profundidad, los Security Groups:}
\begin{enumerate}[label=\alph*)]
    \item Son \textit{stateless} y se aplican a nivel de subred.
    \item Son \textit{stateful} y se aplican a nivel de instancia.
    \item Solo controlan tráfico saliente, no entrante.
    \item Reemplazan completamente la necesidad de NACLs.
\end{enumerate}

\item \textbf{Un bastion host se utiliza principalmente para:}
\begin{enumerate}[label=\alph*)]
    \item Servir contenido web público a los clientes.
    \item Proporcionar un punto de acceso SSH seguro a instancias en subredes privadas.
    \item Actuar como NAT Gateway gestionado.
    \item Funcionar como base de datos central.
\end{enumerate}

\item \textbf{¿Cuál es la ventaja de distribuir subredes en múltiples zonas de disponibilidad (AZs)?}
\begin{enumerate}[label=\alph*)]
    \item Reducir el costo de instancias EC2.
    \item Incrementar la capacidad de almacenamiento de S3.
    \item Mejorar la alta disponibilidad y tolerancia a fallos.
    \item Eliminar la necesidad de balanceadores de carga.
\end{enumerate}

\item \textbf{VPC Peering en el contexto del proyecto integrador se utiliza (a nivel teórico) para:}
\begin{enumerate}[label=\alph*)]
    \item Conectar dos VPCs de forma privada usando la red de AWS.
    \item Conectar directamente una VPC con internet.
    \item Reemplazar un NAT Gateway.
    \item Crear una base de datos replicada entre regiones.
\end{enumerate}

\item \textbf{¿Cuál de las siguientes afirmaciones sobre VPC Flow Logs es correcta?}
\begin{enumerate}[label=\alph*)]
    \item Solo registran tráfico aceptado (ACCEPT).
    \item Pueden enviar logs a CloudWatch Logs o a S3.
    \item Solo funcionan en subredes públicas.
    \item Solo registran tráfico de salida a internet.
\end{enumerate}

\item \textbf{En el análisis de costos, es importante:}
\begin{enumerate}[label=\alph*)]
    \item Ignorar el Free Tier y asumir siempre el peor escenario.
    \item Conocer qué servicios tienen ofertas gratuitas y cuáles generan cargos desde el inicio.
    \item Usar todos los servicios posibles sin importar el costo.
    \item Evitar el uso de CloudWatch para reducir costos.
\end{enumerate}

\item \textbf{Un CloudWatch Dashboard bien diseñado para esta arquitectura debe incluir, como mínimo:}
\begin{enumerate}[label=\alph*)]
    \item Solo gráficos de CPU de instancias internas.
    \item Gráficos de tráfico de red, métricas de conexiones rechazadas y estado de alarmas relevantes.
    \item Únicamente un gráfico con el costo mensual estimado.
    \item Un listado de usuarios IAM sin métricas.
\end{enumerate}

\end{enumerate}

\subsection{Preguntas Verdadero/Falso}

\begin{enumerate}[label=\textbf{VF\arabic*.}]

\item \textbf{En el proyecto integrador, se puede lograr una implementación básica completamente dentro del Free Tier si se controla el número de instancias y la duración de uso.}

\item \textbf{El diseño teórico de alta disponibilidad con balanceadores, Auto Scaling y NAT Gateway no debe implementarse en este laboratorio para evitar costos, pero sí debe documentarse.}

\item \textbf{Una buena práctica es permitir acceso SSH a todas las instancias directamente desde internet para simplificar la administración.}

\end{enumerate}

\subsection{Escenarios Prácticos}

\begin{enumerate}[label=\textbf{E\arabic*.}]

\item \textbf{Escenario 1: Cambio de requisitos de seguridad}

AcmeCorp decide que el servidor web no debe permitir acceso SSH desde ninguna IP pública, y que toda administración debe hacerse únicamente desde el bastion host.

\textbf{Pregunta:} ¿Qué cambios realizarías en los Security Groups y flujos de acceso para cumplir este requisito sin romper la arquitectura?

\item \textbf{Escenario 2: Crecimiento de la empresa}

La empresa crece y ahora requiere que la aplicación web soporte el doble de usuarios, manteniendo alta disponibilidad. El presupuesto de producción permite agregar algunos servicios de pago.

\textbf{Pregunta:} ¿Qué modificaciones propondrías a la arquitectura actual (componentes adicionales y cambios) para soportar esta nueva carga, manteniendo buenas prácticas de seguridad y monitoreo?

\end{enumerate}

\newpage

\subsection{Respuestas del Cuestionario}

\subsubsection*{Selección Múltiple}

\begin{enumerate}[leftmargin=*]
    \item \textbf{b)} Diseñar una arquitectura corporativa completa integrando los conceptos de los labs 1–7.
    \item \textbf{c)} Instancias internas de aplicación y bases de datos sin IP pública.
    \item \textbf{c)} NAT Gateway con tráfico de salida a internet (tiene costo).
    \item \textbf{b)} Son \textit{stateful} y se aplican a nivel de instancia.
    \item \textbf{b)} Proporcionar un punto de acceso SSH seguro a instancias en subredes privadas.
    \item \textbf{c)} Mejorar la alta disponibilidad y tolerancia a fallos.
    \item \textbf{a)} Conectar dos VPCs de forma privada usando la red de AWS.
    \item \textbf{b)} Pueden enviar logs a CloudWatch Logs o a S3.
    \item \textbf{b)} Conocer qué servicios tienen ofertas gratuitas y cuáles generan cargos desde el inicio.
    \item \textbf{b)} Gráficos de tráfico de red, métricas de conexiones rechazadas y estado de alarmas relevantes.
\end{enumerate}

\subsubsection*{Verdadero/Falso}

\begin{enumerate}[leftmargin=*]
    \item \textbf{Verdadero.} Controlando el número de instancias y el tiempo de ejecución, es posible mantenerse en el Free Tier.
    \item \textbf{Verdadero.} Se recomienda documentar el diseño completo pero no implementarlo para evitar cargos.
    \item \textbf{Falso.} La buena práctica es limitar el acceso SSH mediante bastion host y rangos de IP confiables.
\end{enumerate}

\subsubsection*{Guía para Escenarios}

\textbf{Escenario 1:}
\begin{itemize}[leftmargin=*]
    \item Modificar \texttt{SG-Web} para:
    \begin{itemize}
        \item Eliminar cualquier regla de inbound SSH (22) desde internet.
        \item Permitir únicamente HTTP/HTTPS desde \texttt{0.0.0.0/0}.
    \end{itemize}
    \item Asegurar que \texttt{SG-Bastion} pueda acceder por SSH al servidor web:
    \begin{itemize}
        \item Agregar en \texttt{SG-Web} una regla SSH con origen = \texttt{SG-Bastion}.
    \end{itemize}
    \item Flujo de acceso:
    \begin{itemize}
        \item Administrador $\rightarrow$ bastion (SSH).
        \item Bastion $\rightarrow$ servidor web (SSH) y servidor interno.
    \end{itemize}
\end{itemize}

\textbf{Escenario 2:}
\begin{itemize}[leftmargin=*]
    \item Propuestas:
    \begin{itemize}
        \item Agregar un Application Load Balancer frente a la capa web.
        \item Crear un Auto Scaling Group con mínimo 2 instancias web en subredes públicas (AZ1 y AZ2).
        \item Usar NAT Gateway para permitir salida segura a internet desde subredes privadas (actualizaciones, dependencias).
        \item Considerar RDS Multi-AZ para la capa de datos.
        \item Extender el monitoreo:
        \begin{itemize}
            \item Métricas de ALB (RequestCount, errores 4xx/5xx).
            \item Alarmas sobre tiempos de respuesta y tasa de errores.
            \item Dashboards adicionales para la capa de aplicación y base de datos.
        \end{itemize}
    \end{itemize}
    \item Mantener buenas prácticas:
    \begin{itemize}
        \item SGs y NACLs bien definidos.
        \item IAM con mínimo privilegio.
        \item MFA en cuentas administrativas.
    \end{itemize}
\end{itemize}

\newpage

% ================== CONCLUSIONES ==================
\section{Conclusiones}

El Laboratorio 8 representa la culminación de la serie de prácticas de redes en AWS, integrando todos los conceptos abordados en los laboratorios previos en una \textbf{arquitectura corporativa coherente, segura y monitoreada}. A través del diseño y la implementación básica de la VPC \texttt{AcmeCorp-VPC}, subredes públicas y privadas, instancias EC2, Security Groups multicapa, VPC Flow Logs y CloudWatch, el estudiante ha experimentado el ciclo completo de construcción de una red en la nube, desde la planificación hasta la observabilidad.

Los principales aprendizajes incluyen:

\begin{itemize}[leftmargin=*]
    \item La importancia de un \textbf{buen diseño de direccionamiento IP} y separación lógica en subredes públicas y privadas.
    \item La aplicación práctica del \textbf{principio de mínimo privilegio} y la \textbf{defensa en profundidad}, combinando Security Groups, NACLs e IAM.
    \item La relevancia de contar con \textbf{mecanismos de administración seguros}, como el uso de un bastion host para acceder a recursos privados.
    \item La necesidad de \textbf{monitorear el tráfico de red} mediante VPC Flow Logs, métricas derivadas y alarmas para detectar comportamientos anómalos.
    \item La capacidad de diferenciar entre \textbf{implementaciones sin costo (Free Tier)} y \textbf{diseños de producción} que incluyen componentes adicionales como ALB, NAT Gateway, Auto Scaling y RDS.
\end{itemize}

Este proyecto integrador ha demostrado que es posible construir, con recursos mínimos, una arquitectura que ya refleja las buenas prácticas de la industria en términos de seguridad, disponibilidad, escalabilidad y observabilidad. En un entorno real de producción, esta base podría evolucionar hacia una solución aún más robusta incorporando servicios gestionados, automatización y despliegues continuos, sin perder los principios fundamentales aprendidos.

En síntesis, el estudiante no solo ha aprendido a \textbf{usar servicios individuales de AWS}, sino a \textbf{pensar en arquitecturas completas}, evaluando decisiones técnicas, costos, riesgos y mecanismos de operación, capacidades esenciales para cualquier profesional que diseñe redes y sistemas en la nube.

\newpage

% ================== REFERENCIAS ==================
\section{Referencias}

\subsection{Documentación Oficial de AWS}

\begin{enumerate}[leftmargin=*]
    \item AWS Documentation - Página principal \\
    \url{https://docs.aws.amazon.com/}
    \item Amazon VPC User Guide \\
    \url{https://docs.aws.amazon.com/vpc/latest/userguide/}
    \item Amazon EC2 User Guide \\
    \url{https://docs.aws.amazon.com/ec2/}
    \item Amazon CloudWatch Documentation \\
    \url{https://docs.aws.amazon.com/cloudwatch/}
    \item VPC Flow Logs \\
    \url{https://docs.aws.amazon.com/vpc/latest/userguide/flow-logs.html}
    \item AWS Identity and Access Management (IAM) \\
    \url{https://docs.aws.amazon.com/IAM/latest/UserGuide/}
    \item AWS Free Tier \\
    \url{https://aws.amazon.com/free/}
\end{enumerate}

\subsection{AWS Well-Architected Framework}

\begin{enumerate}[leftmargin=*]
    \item AWS Well-Architected Framework \\
    \url{https://aws.amazon.com/architecture/well-architected/}
    \item Security Pillar - AWS Well-Architected Framework \\
    \url{https://docs.aws.amazon.com/wellarchitected/latest/security-pillar/welcome.html}
    \item Reliability Pillar - AWS Well-Architected Framework \\
    \url{https://docs.aws.amazon.com/wellarchitected/latest/reliability-pillar/welcome.html}
\end{enumerate}

\subsection{Bibliografía Recomendada}

\begin{enumerate}[leftmargin=*]
    \item Wittig, A., \& Wittig, M. (2018). \textit{Amazon Web Services in Action} (2nd ed.). Manning Publications.
    \item Erl, T., Mahmood, Z., \& Puttini, R. (2013). \textit{Cloud Computing: Concepts, Technology \& Architecture}. Prentice Hall.
    \item NIST Special Publication 800-145. (2011). \textit{The NIST Definition of Cloud Computing}. \\
    \url{https://nvlpubs.nist.gov/nistpubs/Legacy/SP/nistspecialpublication800-145.pdf}
\end{enumerate}

\vspace{1cm}

\textbf{Nota:} Las URLs de la documentación oficial de AWS fueron verificadas al momento de la elaboración de este laboratorio. Dado que AWS actualiza constantemente sus servicios y guías, se recomienda consultar siempre la versión más reciente en el sitio oficial.

\end{document}
