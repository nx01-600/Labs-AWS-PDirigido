\documentclass[12pt,a4paper]{article}

% Paquetes necesarios
\usepackage[utf8]{inputenc}
\usepackage[spanish]{babel}
\usepackage{graphicx}
\usepackage{xcolor}
\usepackage{hyperref}
\usepackage{geometry}
\usepackage{fancyhdr}
\usepackage{tabularx}
\usepackage{longtable}
\usepackage{enumitem}
\usepackage{amsmath}

% Configuración de página
\geometry{
    left=2cm,
    right=2cm,
    top=2.5cm,
    bottom=2.5cm
}

% Configuración de encabezado y pie de página
\pagestyle{fancy}
\fancyhf{}
\fancyhead[L]{Banco de Cuestionarios - AWS Redes}
\fancyhead[R]{Lab \#\thesection}
\fancyfoot[C]{\thepage}

% Configuración de hipervínculos
\hypersetup{
    colorlinks=true,
    linkcolor=blue,
    filecolor=magenta,      
    urlcolor=cyan,
}

% Colores para respuestas
\definecolor{correcta}{RGB}{0,128,0}
\definecolor{incorrecta}{RGB}{180,180,180}

% Comando para marcar respuesta correcta
\newcommand{\correcta}[1]{\textcolor{correcta}{\textbf{#1} ✓}}
\newcommand{\incorrecta}[1]{\textcolor{incorrecta}{#1}}

\begin{document}

% ================== PORTADA ==================
\begin{titlepage}
    \centering
    \vspace{2cm}
    {\Huge\bfseries Banco de Cuestionarios\par}
    \vspace{0.5cm}
    {\Large\bfseries Laboratorios Virtuales de Redes en AWS\par}
    \vspace{2cm}
    
    {\large\textbf{Proyecto:}\par}
    {\large Laboratorios Virtuales de Redes en AWS para el\par}
    {\large Fortalecimiento de Competencias en Redes de Nueva Generación\par}
    \vspace{1.5cm}
    
    {\large\textbf{Estudiantes:}\par}
    {\large Nicolás Carreño Tascón\par}
    {\large Juan Manuel Canchala Jiménez\par}
    \vspace{1cm}
    
    {\large\textbf{Director:}\par}
    {\large Carlos Olarte\par}
    \vspace{1.5cm}
    
    {\large\textbf{Total de Preguntas:} 120+\par}
    {\large\textbf{Laboratorios Cubiertos:} 12\par}
    \vspace{1cm}
    
    {\large \today\par}
\end{titlepage}

% ================== ÍNDICE ==================
\tableofcontents
\newpage

% ================== INSTRUCCIONES ==================
\section*{Instrucciones de Uso}
\addcontentsline{toc}{section}{Instrucciones de Uso}

\subsection*{Para Estudiantes}
\begin{itemize}
    \item Cada laboratorio tiene su propio conjunto de preguntas
    \item Las preguntas evalúan conocimiento teórico y práctico
    \item Se recomienda completar el cuestionario después de cada lab
    \item Las respuestas correctas están marcadas en \textcolor{correcta}{verde con ✓}
\end{itemize}

\subsection*{Para Instructores}
\begin{itemize}
    \item Preguntas organizadas por laboratorio y tipo
    \item Fácilmente adaptables para Kahoot, Quizizz, o Google Forms
    \item Niveles de dificultad indicados: B (Básico), I (Intermedio), A (Avanzado)
    \item Explicaciones incluidas para respuestas incorrectas comunes
\end{itemize}

\subsection*{Tipos de Preguntas}
\begin{enumerate}
    \item \textbf{Opción Múltiple:} Una respuesta correcta entre 4 opciones
    \item \textbf{Verdadero/Falso:} Afirmaciones para validar conceptos
    \item \textbf{Escenarios:} Situaciones prácticas con soluciones
    \item \textbf{Troubleshooting:} Identificar y resolver problemas
\end{enumerate}

\newpage

% ================== LAB 1: INTRODUCCIÓN A AWS ==================
\section{Laboratorio 1: Introducción a AWS y Configuración Inicial}

\subsection{Preguntas de Opción Múltiple}

\textbf{Pregunta 1.1} [B] - ¿Qué significa "AWS"?
\begin{enumerate}[label=\Alph*)]
    \item \incorrecta{Amazon Web System}
    \item \correcta{Amazon Web Services}
    \item \incorrecta{Automated Web Services}
    \item \incorrecta{Amazon Wireless Services}
\end{enumerate}
\textit{Explicación: AWS son las siglas de Amazon Web Services, la plataforma de servicios en la nube de Amazon.}

\vspace{0.5cm}

\textbf{Pregunta 1.2} [B] - ¿Cuál es el propósito principal de AWS IAM?
\begin{enumerate}[label=\Alph*)]
    \item \incorrecta{Monitorear el rendimiento de las aplicaciones}
    \item \incorrecta{Almacenar datos en la nube}
    \item \correcta{Gestionar identidades y controlar el acceso a recursos de AWS}
    \item \incorrecta{Crear redes virtuales}
\end{enumerate}
\textit{Explicación: IAM (Identity and Access Management) permite gestionar usuarios, grupos, roles y permisos.}

\vspace{0.5cm}

\textbf{Pregunta 1.3} [I] - ¿Cuál es la mejor práctica para la cuenta root de AWS?
\begin{enumerate}[label=\Alph*)]
    \item \incorrecta{Usar la cuenta root para tareas diarias}
    \item \incorrecta{Compartir las credenciales con el equipo}
    \item \correcta{Habilitar MFA y usar usuarios IAM para tareas diarias}
    \item \incorrecta{Desactivar la autenticación de dos factores}
\end{enumerate}
\textit{Explicación: La cuenta root debe protegerse con MFA y usarse solo para tareas administrativas críticas.}

\vspace{0.5cm}

\textbf{Pregunta 1.4} [B] - ¿Qué es AWS Free Tier?
\begin{enumerate}[label=\Alph*)]
    \item \incorrecta{Un servicio gratuito sin límites de uso}
    \item \correcta{Un programa que ofrece servicios gratuitos con límites específicos durante 12 meses}
    \item \incorrecta{Un descuento del 50\% en todos los servicios}
    \item \incorrecta{Un servicio exclusivo para empresas}
\end{enumerate}

\vspace{0.5cm}

\textbf{Pregunta 1.5} [I] - ¿Cuál de las siguientes es una característica de las Zonas de Disponibilidad (AZ)?
\begin{enumerate}[label=\Alph*)]
    \item \incorrecta{Son ubicaciones geográficas diferentes separadas por países}
    \item \correcta{Son centros de datos aislados dentro de una región con baja latencia}
    \item \incorrecta{Almacenan automáticamente copias de todos los datos}
    \item \incorrecta{Son solo para uso en Europa}
\end{enumerate}

\vspace{0.5cm}

\textbf{Pregunta 1.6} [I] - ¿Qué tipo de política IAM concede permisos completos sobre todos los recursos?
\begin{enumerate}[label=\Alph*)]
    \item \incorrecta{ReadOnlyAccess}
    \item \correcta{AdministratorAccess}
    \item \incorrecta{PowerUserAccess}
    \item \incorrecta{SecurityAudit}
\end{enumerate}

\vspace{0.5cm}

\textbf{Pregunta 1.7} [A] - En el contexto de IAM, ¿qué es un "principal"?
\begin{enumerate}[label=\Alph*)]
    \item \incorrecta{El administrador de la cuenta AWS}
    \item \incorrecta{Una política de seguridad}
    \item \correcta{Una entidad que puede realizar acciones en AWS (usuario, rol, o servicio)}
    \item \incorrecta{El primer usuario creado en la cuenta}
\end{enumerate}

\vspace{0.5cm}

\textbf{Pregunta 1.8} [B] - ¿Cuál es el número máximo de regiones AWS que puedes usar con una sola cuenta?
\begin{enumerate}[label=\Alph*)]
    \item \incorrecta{Solo 5 regiones}
    \item \incorrecta{10 regiones}
    \item \correcta{Todas las regiones disponibles (sin límite específico)}
    \item \incorrecta{Solo la región donde creaste la cuenta}
\end{enumerate}

\vspace{0.5cm}

\textbf{Pregunta 1.9} [I] - ¿Qué herramienta te permite establecer alertas de facturación en AWS?
\begin{enumerate}[label=\Alph*)]
    \item \incorrecta{AWS IAM}
    \item \incorrecta{AWS Config}
    \item \correcta{AWS Billing Dashboard y CloudWatch}
    \item \incorrecta{AWS CloudTrail}
\end{enumerate}

\vspace{0.5cm}

\textbf{Pregunta 1.10} [A] - ¿Cuál es la diferencia entre una política administrada por AWS y una política administrada por el cliente?
\begin{enumerate}[label=\Alph*)]
    \item \incorrecta{No hay diferencia}
    \item \correcta{Las políticas administradas por AWS son creadas y mantenidas por AWS; las del cliente son personalizadas}
    \item \incorrecta{Las políticas del cliente son más seguras}
    \item \incorrecta{Las políticas administradas por AWS no se pueden usar}
\end{enumerate}

\subsection{Preguntas de Verdadero/Falso}

\textbf{Pregunta 1.11} [B] - AWS es solo un servicio de almacenamiento en la nube.
\begin{itemize}
    \item \incorrecta{Verdadero}
    \item \correcta{Falso}
\end{itemize}
\textit{Explicación: AWS ofrece más de 200 servicios incluyendo cómputo, redes, bases de datos, ML, IoT, etc.}

\vspace{0.5cm}

\textbf{Pregunta 1.12} [B] - MFA añade una capa adicional de seguridad a las cuentas de AWS.
\begin{itemize}
    \item \correcta{Verdadero}
    \item \incorrecta{Falso}
\end{itemize}

\vspace{0.5cm}

\textbf{Pregunta 1.13} [I] - Un usuario IAM puede pertenecer a múltiples grupos.
\begin{itemize}
    \item \correcta{Verdadero}
    \item \incorrecta{Falso}
\end{itemize}

\vspace{0.5cm}

\textbf{Pregunta 1.14} [I] - Las credenciales de acceso (Access Keys) deben compartirse abiertamente en el equipo.
\begin{itemize}
    \item \incorrecta{Verdadero}
    \item \correcta{Falso}
\end{itemize}
\textit{Explicación: Las credenciales son personales y nunca deben compartirse. Cada usuario debe tener las suyas.}

\vspace{0.5cm}

\textbf{Pregunta 1.15} [A] - IAM es un servicio regional, por lo que debes configurarlo en cada región.
\begin{itemize}
    \item \incorrecta{Verdadero}
    \item \correcta{Falso}
\end{itemize}
\textit{Explicación: IAM es un servicio global. Los usuarios y roles se aplican a todas las regiones.}

\subsection{Preguntas de Escenario}

\textbf{Pregunta 1.16} [I] - Tu empresa necesita que un grupo de desarrolladores tenga acceso de solo lectura a los recursos de S3 pero no a EC2. ¿Cuál es la mejor forma de implementar esto?
\begin{enumerate}[label=\Alph*)]
    \item \incorrecta{Dar a cada desarrollador acceso administrativo completo}
    \item \correcta{Crear un grupo IAM con política de solo lectura a S3 y añadir los desarrolladores al grupo}
    \item \incorrecta{Compartir la cuenta root con el equipo}
    \item \incorrecta{Crear una política personalizada para cada desarrollador individualmente}
\end{enumerate}

\vspace{0.5cm}

\textbf{Pregunta 1.17} [A] - Has detectado un cargo inesperado en tu cuenta AWS. ¿Cuál es el primer paso que deberías tomar?
\begin{enumerate}[label=\Alph*)]
    \item \incorrecta{Eliminar la cuenta inmediatamente}
    \item \incorrecta{Cambiar a otro proveedor de nube}
    \item \correcta{Revisar el Billing Dashboard para identificar el servicio y región causante del cargo}
    \item \incorrecta{Llamar al soporte sin investigar}
\end{enumerate}

\vspace{0.5cm}

\textbf{Pregunta 1.18} [I] - Necesitas dar acceso temporal a un consultor externo para revisar la configuración de seguridad. ¿Qué deberías hacer?
\begin{enumerate}[label=\Alph*)]
    \item \incorrecta{Compartir tu contraseña personal}
    \item \correcta{Crear un usuario IAM temporal con permisos limitados y MFA, eliminarlo después}
    \item \incorrecta{Dar acceso completo administrativo}
    \item \incorrecta{No dar ningún acceso}
\end{enumerate}

\newpage

% ================== LAB 2: VPC FUNDAMENTOS ==================
\section{Laboratorio 2: Fundamentos de Amazon VPC}

\subsection{Preguntas de Opción Múltiple}

\textbf{Pregunta 2.1} [B] - ¿Qué significa VPC?
\begin{enumerate}[label=\Alph*)]
    \item \incorrecta{Virtual Personal Computer}
    \item \correcta{Virtual Private Cloud}
    \item \incorrecta{Virtual Public Connection}
    \item \incorrecta{Virtual Protected Container}
\end{enumerate}

\vspace{0.5cm}

\textbf{Pregunta 2.2} [I] - ¿Cuál es el rango CIDR máximo que puedes asignar a una VPC?
\begin{enumerate}[label=\Alph*)]
    \item \incorrecta{/8}
    \item \incorrecta{/12}
    \item \correcta{/16}
    \item \incorrecta{/24}
\end{enumerate}
\textit{Explicación: Una VPC puede tener desde /16 (65,536 IPs) hasta /28 (16 IPs).}

\vspace{0.5cm}

\textbf{Pregunta 2.3} [B] - ¿Qué bloque CIDR es válido para uso privado según RFC 1918?
\begin{enumerate}[label=\Alph*)]
    \item \incorrecta{8.8.8.0/24}
    \item \correcta{10.0.0.0/8}
    \item \incorrecta{200.100.50.0/24}
    \item \incorrecta{130.0.0.0/16}
\end{enumerate}
\textit{Explicación: Los rangos privados son: 10.0.0.0/8, 172.16.0.0/12, y 192.168.0.0/16}

\vspace{0.5cm}

\textbf{Pregunta 2.4} [I] - En una VPC, ¿cuántas direcciones IP se reservan automáticamente en cada subnet?
\begin{enumerate}[label=\Alph*)]
    \item \incorrecta{3}
    \item \incorrecta{4}
    \item \correcta{5}
    \item \incorrecta{6}
\end{enumerate}
\textit{Explicación: AWS reserva 5 IPs: red, router, DNS, futuro uso, y broadcast.}

\vspace{0.5cm}

\textbf{Pregunta 2.5} [I] - ¿Qué determina si una subnet es pública o privada?
\begin{enumerate}[label=\Alph*)]
    \item \incorrecta{El bloque CIDR que usa}
    \item \incorrecta{Si tiene instancias EC2}
    \item \correcta{Si tiene una ruta a un Internet Gateway en su tabla de ruteo}
    \item \incorrecta{Su zona de disponibilidad}
\end{enumerate}

\vspace{0.5cm}

\textbf{Pregunta 2.6} [A] - Si tienes una VPC con CIDR 10.0.0.0/16, ¿cuál sería un CIDR válido para una subnet?
\begin{enumerate}[label=\Alph*)]
    \item \incorrecta{10.1.0.0/24}
    \item \correcta{10.0.1.0/24}
    \item \incorrecta{172.16.0.0/24}
    \item \incorrecta{10.0.0.0/8}
\end{enumerate}

\vspace{0.5cm}

\textbf{Pregunta 2.7} [B] - ¿Cuántas VPCs puedes crear por región en una cuenta AWS (límite por defecto)?
\begin{enumerate}[label=\Alph*)]
    \item \incorrecta{1}
    \item \correcta{5}
    \item \incorrecta{10}
    \item \incorrecta{Ilimitadas}
\end{enumerate}
\textit{Nota: Este límite se puede aumentar contactando a AWS Support.}

\vspace{0.5cm}

\textbf{Pregunta 2.8} [I] - ¿Qué componente controla el tráfico que entra y sale de una subnet?
\begin{enumerate}[label=\Alph*)]
    \item \incorrecta{Security Group}
    \item \correcta{Network ACL (NACL)}
    \item \incorrecta{Internet Gateway}
    \item \incorrecta{Route Table}
\end{enumerate}

\vspace{0.5cm}

\textbf{Pregunta 2.9} [A] - ¿Cuál es la diferencia principal entre IPv4 e IPv6 en una VPC?
\begin{enumerate}[label=\Alph*)]
    \item \incorrecta{IPv6 es más lento}
    \item \incorrecta{IPv6 no está disponible en AWS}
    \item \correcta{IPv6 proporciona direcciones públicas únicas globalmente y mayor espacio de direcciones}
    \item \incorrecta{IPv6 solo funciona en subnets privadas}
\end{enumerate}

\vspace{0.5cm}

\textbf{Pregunta 2.10} [I] - ¿Puedes cambiar el bloque CIDR de una VPC después de crearla?
\begin{enumerate}[label=\Alph*)]
    \item \incorrecta{Sí, sin limitaciones}
    \item \correcta{Puedes agregar CIDRs secundarios, pero no modificar el primario}
    \item \incorrecta{No, nunca puedes modificarlo}
    \item \incorrecta{Solo en los primeros 30 días}
\end{enumerate}

\subsection{Preguntas de Verdadero/Falso}

\textbf{Pregunta 2.11} [B] - Una VPC abarca múltiples regiones de AWS.
\begin{itemize}
    \item \incorrecta{Verdadero}
    \item \correcta{Falso}
\end{itemize}
\textit{Explicación: Una VPC es específica de una región, pero puede abarcar múltiples AZs dentro de esa región.}

\vspace{0.5cm}

\textbf{Pregunta 2.12} [I] - Puedes tener subnets en diferentes zonas de disponibilidad dentro de la misma VPC.
\begin{itemize}
    \item \correcta{Verdadero}
    \item \incorrecta{Falso}
\end{itemize}

\vspace{0.5cm}

\textbf{Pregunta 2.13} [I] - Los bloques CIDR de las subnets dentro de una VPC pueden superponerse.
\begin{itemize}
    \item \incorrecta{Verdadero}
    \item \correcta{Falso}
\end{itemize}
\textit{Explicación: Los CIDRs de subnets no pueden superponerse dentro de la misma VPC.}

\vspace{0.5cm}

\textbf{Pregunta 2.14} [B] - Cada VPC viene con una tabla de ruteo principal por defecto.
\begin{itemize}
    \item \correcta{Verdadero}
    \item \incorrecta{Falso}
\end{itemize}

\vspace{0.5cm}

\textbf{Pregunta 2.15} [A] - El DHCP en AWS VPC es configurable y obligatorio para todas las instancias.
\begin{itemize}
    \item \correcta{Verdadero}
    \item \incorrecta{Falso}
\end{itemize}
\textit{Explicación: AWS proporciona DHCP automáticamente, pero puedes configurar opciones personalizadas.}

% ================== PLACEHOLDER PARA MÁS LABS ==================
\section{Laboratorio 3: Conectividad a Internet en VPC}
\textit{[15 preguntas sobre Internet Gateway, NAT Gateway, Elastic IP...]}

\section{Laboratorio 4: Amazon EC2 y Security Groups}
\textit{[15 preguntas sobre EC2, Security Groups...]}

\section{Laboratorio 5: Seguridad Avanzada - Network ACLs y VPC Flow Logs}
\textit{[15 preguntas sobre NACLs, Flow Logs, análisis de tráfico...]}

\section{Laboratorio 6: VPC Peering}
\textit{[15 preguntas sobre VPC Peering, conectividad entre VPCs...]}

\section{Laboratorio 7: Monitoreo y CloudWatch}
\textit{[15 preguntas sobre CloudWatch, alarmas, métricas, Reachability Analyzer...]}

\section{Laboratorio 8: Proyecto Integrador - Arquitectura 3-Tier}
\textit{[20 preguntas de escenarios complejos e integración...]}

% ================== SECCIÓN DE RESPUESTAS ==================
\newpage
\section*{Guía de Respuestas Rápidas}
\addcontentsline{toc}{section}{Guía de Respuestas Rápidas}

\subsection*{Laboratorio 1}
1.1: B | 1.2: C | 1.3: C | 1.4: B | 1.5: B | 1.6: B | 1.7: C | 1.8: C | 1.9: C | 1.10: B \\
1.11: F | 1.12: V | 1.13: V | 1.14: F | 1.15: F | 1.16: B | 1.17: C | 1.18: B

\subsection*{Laboratorio 2}
2.1: B | 2.2: C | 2.3: B | 2.4: C | 2.5: C | 2.6: B | 2.7: B | 2.8: B | 2.9: C | 2.10: B \\
2.11: F | 2.12: V | 2.13: F | 2.14: V | 2.15: V

\textit{[Continuar con el resto de laboratorios...]}

% ================== ANEXOS ==================
\newpage
\appendix
\section{Anexo A: Adaptación para Kahoot}

\subsection*{Consejos para importar a Kahoot}
\begin{itemize}
    \item Usar opción múltiple (máximo 4 opciones)
    \item Tiempo recomendado: 30-45 segundos por pregunta
    \item Marcar la dificultad en la plataforma
    \item Agregar imágenes cuando sea posible
\end{itemize}

\subsection*{Formato de importación}
\begin{verbatim}
Pregunta, Opción1, Opción2, Opción3, Opción4, RespuestaCorrecta, Tiempo
\end{verbatim}

\section{Anexo B: Rúbricas de Evaluación}

\begin{table}[h]
\centering
\begin{tabular}{|l|c|c|c|}
\hline
\textbf{Nivel} & \textbf{Porcentaje} & \textbf{Calificación} & \textbf{Descripción} \\
\hline
Excelente & 90-100\% & A & Dominio completo \\
Bueno & 80-89\% & B & Buen entendimiento \\
Aceptable & 70-79\% & C & Conocimiento básico \\
Insuficiente & <70\% & D & Requiere repasar \\
\hline
\end{tabular}
\caption{Escala de evaluación}
\end{table}

\section{Anexo C: Recursos Complementarios}

\subsection*{Plataformas de Cuestionarios}
\begin{itemize}
    \item \textbf{Kahoot:} Ideal para sesiones en vivo, gamificado
    \item \textbf{Quizizz:} Permite ritmo individual, más detallado
    \item \textbf{Google Forms:} Flexible, fácil análisis de resultados
    \item \textbf{Moodle Quiz:} Para integración con LMS
\end{itemize}

\end{document}
